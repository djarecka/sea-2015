\documentclass{beamer}
\setbeamercolor{alerted text}{fg=magenta}
\usetheme{Boulder}
%34,139,34
%\definecolor{bgcolor}{rgb}{34/256,139/256,34/256}
\setbeamertemplate{blocks}[rounded][shadow=true]
\setbeamercolor{block title}{fg=white,bg=structure}
\setbeamercolor{block body}{bg=white}
\author[Dorota Jarecka]{Dorota Jarecka, Sylwester Arabas, Anna Jaruga}
\title{TODO}
\date{2014 April 7\textsuperscript{th}\\UCAR SEA Software Engineering Conference 2014\\Boulder, CO, USA}
\institute{University of Warsaw}

\usepackage{tikz,tikz-qtree}
\usepackage[utf8]{inputenc}


% listings                                                                              
\usepackage{fancyvrb,relsize}
%\fvset{frame=single,framerule=.2mm,fontsize=\relscale{1.2}}
\fvset{fontsize=\relscale{1.}}
\input{lst/pygments}

\newcommand*\FancyVerbStartString{}
\newcommand*\FancyVerbStopString{}

\newcommand{\codepyt}[3]{%                                                              
%  \refstepcounter{Listing}%                                                             
%  \fvset{label=Listing~\theListing}%                                                    
%  \fvset{gobble=#3}%                                                                    
  \renewcommand*\FancyVerbStartString{\PY{c}{\PYZsh{}\PYZlt{}listing\PYZhy{}#2\PYZgt{}}}%                                                                                      
  \renewcommand*\FancyVerbStopString{\PY{c}{\PYZsh{}\PYZlt{}/listing\PYZhy{}#2\PYZgt{}}}%                                                                                      
  \input{#1}%                                                                           
%  \vspace{-.5em}%                                                                      
}

\newcommand{\codetxt}[4]{%                                                           
  \refstepcounter{Listing}%                                                         
  \fvset{label=Listing~\theListing#4}%
  \fvset{fontsize=\relscale{.7}}                                        
  \fvset{gobble=#3}%                                                                 
  \renewcommand*\FancyVerbStartString{\PYZsh{}\PYZlt{}listing\PYZhy{}#2\PYZgt{}}%  
  \renewcommand*\FancyVerbStopString{\PYZsh{}\PYZlt{}/listing\PYZhy{}#2\PYZgt{}}% 
  \input{#1}%                                            
  \vspace{-1.5em}%                                       
}



\usepackage{float}
\newfloat{Listing}{hbtp}{lop}


\newcommand{\ftitle}[1]{\frametitle{#1\\\hrule}\vspace{-2em}}

\AtBeginSection[]{%
  \begin{frame}<beamer>
    \ftitle{talk outline}
    \only<1>{\tableofcontents}
    \only<2>{\tableofcontents[currentsection]}
  \end{frame}
  \addtocounter{framenumber}{-1}% If you don't want them to affect the slide number
}

\begin{document}
  \begin{frame}
    \maketitle
  \end{frame}

  \section{Introduction: testing and py.test}


  \begin{frame}[fragile]
    \ftitle{software testing - motivation}
    \uncover<1->{
      {\bf why testing a software?}
%  \begin{Listing}
%    \codepyt{lst/test}{1}{0}
%    \label{lst:git_revision}
%  \end{Listing}

       \vspace{.5em}
      \begin{itemize}
        \uncover<2->{\item{mistakes happend and always will}\\{\color{gray}(don't deny!)}\\}
        \vspace{.5em}
        \uncover<3->{\alert{$\leadsto$ guard agains them}\\}
        \vspace{.5em}
        \uncover<3->{\alert{$\leadsto$ raise your confidence during development}\\}
          \vspace{.5em}
	\uncover<4->{\item{makes you think about desirable output}\\}
          \vspace{.5em}
	\uncover<5->{\alert{$\leadsto$ helps you to write a better code}\\}
          \vspace{.5em}
        \uncover<4->{\item{improves readability of your code}\\}
          \vspace{.5em}
        \uncover<5->{\alert{$\leadsto$ helps to reuse your code}\\}
%allow for later changes??
% dodac when testing??
      \end{itemize}
    }
\end{frame}

  \begin{frame}
    \ftitle{software testing - types}
   \begin{description}
      \item[unittest]{testing isolated parts of the code}
         \vspace{1em}
     \item[integration]{checking if components cooperate}
         \vspace{1em}
     \item[functional]{checking if code works in an environment}
    \end{description}
   \vspace{2.em}
        \uncover<2->{\alert{$\leadsto$ We will concentrate on unittests}\\}
  \end{frame}

  \begin{frame}[fragile]
      \ftitle{unit tests and assert statement}
   \uncover<1>{A simple example - testing division by two}
   \only<1-5,8>{ \codepyt{lst/test_asserts}{1}{0}}
     \only<2,3,4,5,6,8>{ \begin{itemize}
   \item{\only<2-3>{the simplest assert statement\\
            \codepyt{lst/test_asserts}{2}{0}}
      \only<4-5>{the assert statement with message\\
          \codepyt{lst/test_asserts}{3}{0}}
      \only<6>{using Unittest built-in library\\
            \codepyt{lst/test_asserts}{4}{0}}
      \only<8>{using py.test\\
            \codepyt{lst/test_asserts}{5}{0}}}
      \end{itemize}}
    \only<3>{\codetxt{lst/output_asserts}{1}{0}{}}
    \only<5>{\codetxt{lst/output_asserts}{2}{0}{}}
    \only<7>{\codetxt{lst/output_asserts}{3}{0}{}}
    \only<9>{\codetxt{lst/output_asserts}{4}{0}{}}
% TODO dolozyc poprawiona wersje i output!

%    \codeout{lst/output_asserts.txt}
\end{frame}

  \begin{frame}
    \ftitle{py.test}
     \begin{itemize}
      \item{it's easy to get started}
         \vspace{1em}
     \item{straightforward asserting with the assert statement}
         \vspace{1em}
     \item{helpful traceback and failing assertion reporting}
         \vspace{1em}
     \item{automatic test discovery}
    \end{itemize}
   \vspace{2.em}
        \uncover<2->{\alert{ \url{http://pytest.org/latest/index.html}}}
  \end{frame}

 \section{py.test - examples}

 \begin{frame}[fragile]
      \ftitle{more about assert statement}
   Potential temeperature
    \only<1-3>{$$ \theta = T \left(\frac{p_0}{p} \right)^{(R_d/cp)}  $$}
    \only<4->{\codepyt{lst/test_asserts_temp}{1}{0}}
 
  LC
 \begin{itemize}
   \uncover<2->{\item{testing input}
   \codepyt{lst/test_asserts_temp}{5}{0}}
   \uncover<3->{\item{testing output}
   \codepyt{lst/test_asserts_temp}{3}{0}}
 \end{itemize}


\end{frame}

\section{py.test - options and...}

\section{py.test - more advanced features}

  \frame<-4>[label=tabelka]{
    \ftitle{software design and researchers' productivity}
    \begin{center}\large
      \begin{columns}[b]
	\begin{column}{.5\textwidth}
          {\bf users' perspective}
          \vspace{1em}
          \begin{itemize}
            \item<2->{\alert<5>{ease of use}}
            \item<2->{\alert<6>{robustness}}
            \item<2->{\alert<7>{result reproducibility}}
          \end{itemize}
%          \uncover<3->{
%	    $~~~\underbrace{~~~~~~~~~~~~~~~~~~~~~~~~~~~~~~~~~~~~}_{\text{users' perspective}}$
%          }
	\end{column}
	\begin{column}{.5\textwidth}
          {\bf developers' perspective}
          \begin{itemize}
            \vspace{2.2em}
            \item<3->{\alert<8>{extendability}}
            \item<3->{\alert<9>{maintainability}}
          \end{itemize}
%          \uncover<4->{
%	    $~~~\underbrace{~~~~~~~~~~~~~~~~~~~~~~~~~~}_{\text{developers' perspective}}$
%          }
	\end{column}
      \end{columns}
    \end{center}
    \vspace{1em}
    \uncover<4->{
      {\Large \color<5->{gray}\alert<4>{researcher = user \& developer}}
    }
  }


\end{document}
